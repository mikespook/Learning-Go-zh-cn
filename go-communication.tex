\epi{"Good communication is as stimulating as black coffee, and just as hard
to sleep after."}{\textsc{ANNE MORROW LINDBERGH}}
\noindent{}In this chapter we are going to look at the building blocks in Go for 
communicating with the outside world.

\section{Files}
Reading from (and writing to) files is easy in Go. This program
only uses the \package{os} package to read data from the file \file{/etc/passwd}.
\lstinputlisting[caption=Reading from a file (unbufferd),label=src:read]{src/file.go}
\showremarks
If you want to use \first{buffered}{buffered} IO there is the
\package{bufio}\index{package!bufio} package:
\lstinputlisting[caption=Reading from a file (bufferd),label=src:bufread]{src/buffile.go}
\showremarks

\subsection{Line by line}
The previous program reads a file in its entirely, but a more common scenario is that
you want to read a file on a line-by-line basis. The following snippet show a way
to do just that:

\begin{lstlisting}
f, _ := os.Open("/etc/passwd")
defer f.Close()
r := bufio.NewReader(f)
s, ok := r.ReadString('\n')     |\coderemark{Read a line from the input}|
// ... \coderemark{\var{s} holds the string, with the \package{strings} package you can parse it}
\end{lstlisting}

A more robust method (but slightly more complicated) is \func{ReadLine}, see the documentation
of the \package{bufio} package.

\section{Command line arguments}
\label{sec:option parsing}
Arguments from the command line are available inside your program via
the string slice \var{os.Args}, provided you have imported the package
\package{os}. The \package{flag} package has a more sophisticated
interface, and also provides a way to parse flags. Take this example
from a DNS query tool:
\begin{lstlisting}
dnssec := flag.Bool("dnssec", false, "Request DNSSEC records") |\longremark{Define a \texttt{bool} flag, %%
\texttt{-dnssec}. The variable must be a pointer otherwise the package can not set its value;}|
port := flag.String("port", "53", "Set the query port")      |\longremark{Idem, but for a \texttt{port} option;}|
flag.Usage = func() {   |\longremark{Slightly redefine the \func{Usage} function, to be a little more verbose;}|
    fmt.Fprintf(os.Stderr, "Usage: %s [OPTIONS] [name ...]\n", os.Args[0])
    flag.PrintDefaults() |\longremark{For every flag given, \func{PrintDefaults} will output the help string;}|
}
flag.Parse()   |\longremark{Parse the flags and fill the variables.}|
\end{lstlisting}
\showremarks

\section{Executing commands}
The \package{exec}\index{package!exec} package has functions to run external commands, and it the premier way to
execute commands from within a Go program. It works by defining a \var{*exec.Cmd} structure for which it
defines a number of methods.
Lets execute \verb|ls -l|:
\begin{lstlisting}
import "exec"

cmd := exec.Command("/bin/ls", "-l")    |\coderemark{Create a \var{*cmd}}|
err := cmd.Run()                        |\coderemark{\func{Run()} it}|
\end{lstlisting}
Capturing standard output from a command is also easy to do:
\begin{lstlisting}
import "exec"

cmd := exec.Command("/bin/ls", "-l")
buf, err := cmd.Ouput()                 |\coderemark{\var{buf} is a (\type{[]byte})}|
\end{lstlisting}

\section{Networking}
All network related types and functions can be found in the package \package{net}. One of the
most important functions in there is \func{Dial}\index{networking!Dial}. When you \func{Dial}
into a remote system the function returns a \var{Conn} interface type, which can be used
to send and receive information. The function \func{Dial} neatly abstracts away the network
family and transport. So IPv4 or IPv6, TCP or UDP can all share a common interface. 

Dialing a remote system (port 80) over TCP, then UDP and lastly TCP over IPv6 looks
like this:\footnote{In case
you are wondering, 192.0.32.10 and 2620:0:2d0:200::10 are \url{www.example.org}.}
\begin{lstlisting}
conn, err := Dial("tcp", "192.0.32.10:80")
conn, err := Dial("udp", "192.0.32.10:80")
conn, err := Dial("tcp", "[2620:0:2d0:200::10]:80") |\coderemark{Brackets are mandatory}|
\end{lstlisting}

And with \var{conn} you can do read/write \todo{dkls}.

\todo{Write echo server}

\section{Netchan: networking and channels}
%%http://blog.golang.org/2010/09/go-concurrency-patterns-timing-out-and.html

\section{Exercises}
\begin{Exercise}[title={进程},difficulty=8]
\label{ex:processes}
\Question\label{ex:processes q1}
编写一个程序,列出所有正在运行的进程,并打印每个进程执行的子进程个数。
输出应当类似:
%% For some reason the spacing in Exercise env. does weird things
\vskip\baselineskip
\begin{display}
Pid 0 has 2 children: [1 2]
Pid 490 has 2 children: [1199 26524]
Pid 1824 has 1 child: [7293]
\end{display}
\vskip\baselineskip
\begin{itemize}
\item{为了获取进程列表,需要得到 \verb|ps -e -opid,ppid,comm| 的输出。输出类似:
\vskip\baselineskip
\begin{display}
  PID  PPID COMMAND
 9024  9023 zsh
19560  9024 ps
\end{display}
\vskip\baselineskip}
\item{如果父进程有一个子进程,就打印 \verb|child|,如果多于一个,就打印 \verb|children|;}
\item{进程列表要按照数字排序,这样就以 pid 0 开始,依次展示。}
\end{itemize}
这里有一个 Perl 版本的程序来帮助上手(或者造成绝对的混乱)。
\lstinputlisting[caption={Processes in Perl}]{ex-communication/src/proc.pl}
\end{Exercise}

\begin{Answer}
\Question 有许多工作需要做。可以将程序分为以下几个部分:
\begin{enumerate}
\item{运行 \verb|ps| 获得输出;}
\item{解析输出并保存每个 PPID 的子 PID;}
\item{排序 PPID 列表;}
\item{打印排序后的列表到屏幕。}
\end{enumerate}
在下面的解法中,选择 \package{container/vector} 保存 PID。"列表"自动增长。

函数 \func{atoi} (19 行到 22 行)被定义为包裹原始的多返回值函数
\func{strconv.Atoi},这样就可以像 45、47 和 50 行那样,作为函数调用时的参数使用。

程序清单:
\lstinputlisting[caption=Go 进程,numbers=right]{ex-communication/src/proc.go}
\end{Answer}


\begin{Exercise}[title={单词和字母统计},difficulty=0]
\label{ex:wc}
\Question\label{ex:wc q1}编写一个从标准输入中读取文本的小程序,
并进行下面的操作:
\begin{enumerate}
\item{计算字符数量(包括空格);}
\item{计算单词数量;}
\item{计算行数。}
\end{enumerate}
换句话说,实现一个 \prog{wc(1)}(参阅本地的手册页面),
然而只需要从标准输入读取。
\end{Exercise}

\begin{Answer}
\Question 下面是 \prog{wc(1)} 的一种实现。
\lstinputlisting[caption=wc(1) 的 Go 实现]{ex-communication/src/wc.go}
\showremarks
\end{Answer}


\begin{Exercise}[title={Uniq},difficulty=0]
\label{ex:Uniq}
\Question\label{ex:Uniq q1} 编写一个 Go 程序模仿 Unix 命令 
\prog{uniq} 的功能。程序应当像下面这样运行,提供一个下面这样的列表:

\begin{display}
'a' 'b' 'a' 'a' 'a' 'c' 'd' 'e' 'f' 'g'
\end{display}

它将打印出没有后续重复的项目:

\begin{display}
'a' 'b' 'a' 'c' 'd' 'e' 'f'
\end{display}
\exdisfix
下面列出的 \ref{src:uniq} 是 Perl 实现的算法。
\lstinputlisting[label=src:uniq,caption=uniq(1) 的 Perl 实现,language=Perl]{ex-communication/src/uniq.pl}

\end{Exercise}

\begin{Answer}
\Question 下面是 uniq 的 Go 实现.
\lstinputlisting[caption=uniq(1) 的 Go 实现]{ex-communication/src/uniq.go}
\end{Answer}


\begin{Exercise}[title={Quine},difficulty=9]
A \emph{Quine} is a program that prints itself.
\label{ex:quine}
\Question\label{ex:quine q1} Write a Quine in Go.
\end{Exercise}

\begin{Answer}
\Question 
The following Quine is from Russ Cox:
\begin{lstlisting}
/* Go quine */
package main
import "fmt"
func main() {
 fmt.Printf("%s%c%s%c\n", q, 0x60, q, 0x60)
}
var q = `/* Go quine */
package main
import "fmt"
func main() {
 fmt.Printf("%s%c%s%c\n", q, 0x60, q, 0x60)
}
var q = `
\end{lstlisting}
\end{Answer}


\begin{Exercise}[title={Number cruncher},difficulty=9]
\label{ex:numbercruncher}
\begin{itemize}
\item{Pick six (6) random numbers from this list:
$$1, 2, 3, 4, 5, 6, 7, 8, 9, 10, 25, 50, 75, 100$$
Numbers may be picked multiple times;}
\item{Pick one (1) random number ($i$) in the range: $1 \ldots 1000$;}
\item{Tell how, by combining the first 6 numbers (or a subset thereof)
with the operators $+$,$-$,$*$ and $/$, you can make $i$;}
\end{itemize}
An example. We have picked the numbers: 1, 6, 7, 8, 8 and 75. And $i$ is
977. This can be done in many different ways, one way is:
$$ ((((1 * 6) * 8) + 75) * 8) - 7 = 977$$ 
or
$$ (8*(75+(8*6)))-(7/1) = 977$$

\Question\label{ex:cruncher q1}
Implement a number cruncher that works like that. Make it print the
solution in a similar format (i.e. output should be infix with
parenthesis) as used above.
\Question\label{ex:cruncher q2}
Calculate \emph{all} possible solutions and show them (or only show how
many there are). In the example above there are 544 ways to do it.
\end{Exercise}

\begin{Answer}
\Question 
The following is one possibility. It uses recursion and backtracking to get
an answer. Is is displayed with a smaller font, because of the size of
the program.
\lstinputlisting[caption=Number cruncher]{ex-communication/src/permrec.go}

\Question
When starting \prog{permrec} we give 977 as the first argument:
\vspace{1em}
\begin{display}
\pr ./permrec 977
1+(((6+7)*75)+(8/8)) = 977  #1
...                         ...
((75+(8*6))*8)-7 = 977      #542
(((75+(8*6))*8)-7)*1 = 977  #543
(((75+(8*6))*8)-7)/1 = 977  #544
\end{display}

\end{Answer}


\cleardoublepage
\section{Answers}
\shipoutAnswer
