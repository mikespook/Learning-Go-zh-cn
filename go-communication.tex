Communication with the outside world:
\begin{itemize}
\item{Files}
\item{Input/Output, Stdin, Stdout}
\item{Networking}
\item{Starting other programs}
\item{Forking}
\end{itemize}

\section{Files}
Reading from (and writing to) files is easy in Go. This program
only uses the \package{os} package to read data from the file \file{/etc/passwd}.
\lstinputlisting[numbers=right,caption=Reading from a file
(unbufferd),label=src:read]{src/file.go}
If you want to use \first{buffered} IO there is the \package{bufio} package:
\lstinputlisting[numbers=right,caption=Reading from a file
(bufferd),label=src:bufread]{src/buffile.go}

On line 12 we create a \type{bufio.Reader} from \var{f} which is of
type \type{*File}. \func{NewReader} expects an \type{io.Reader}, so you
might think this will fail. But it doesn't. An \type{io.Reader} is
defined as:
\begin{lstlisting}
type Reader interface {
    Read(p []byte) (n int, err os.Error)
}
\end{lstlisting}
So \emph{anything} that has such a \func{Read()} function implements this
interface. And from listing \ref{src:read} (line 10) we can see
the \type{*File} indeed does so. 

\section{Exercises}
\begin{Exercise}[title={进程},difficulty=8]
\label{ex:processes}
\Question\label{ex:processes q1}
编写一个程序,列出所有正在运行的进程,并打印每个进程执行的子进程个数。
输出应当类似:
%% For some reason the spacing in Exercise env. does weird things
\vskip\baselineskip
\begin{display}
Pid 0 has 2 children: [1 2]
Pid 490 has 2 children: [1199 26524]
Pid 1824 has 1 child: [7293]
\end{display}
\vskip\baselineskip
\begin{itemize}
\item{为了获取进程列表,需要得到 \verb|ps -e -opid,ppid,comm| 的输出。输出类似:
\vskip\baselineskip
\begin{display}
  PID  PPID COMMAND
 9024  9023 zsh
19560  9024 ps
\end{display}
\vskip\baselineskip}
\item{如果父进程有一个子进程,就打印 \verb|child|,如果多于一个,就打印 \verb|children|;}
\item{进程列表要按照数字排序,这样就以 pid 0 开始,依次展示。}
\end{itemize}
这里有一个 Perl 版本的程序来帮助上手(或者造成绝对的混乱)。
\lstinputlisting[caption={Processes in Perl}]{ex-communication/src/proc.pl}
\end{Exercise}

\begin{Answer}
\Question 有许多工作需要做。可以将程序分为以下几个部分:
\begin{enumerate}
\item{运行 \verb|ps| 获得输出;}
\item{解析输出并保存每个 PPID 的子 PID;}
\item{排序 PPID 列表;}
\item{打印排序后的列表到屏幕。}
\end{enumerate}
在下面的解法中,选择 \package{container/vector} 保存 PID。"列表"自动增长。

函数 \func{atoi} (19 行到 22 行)被定义为包裹原始的多返回值函数
\func{strconv.Atoi},这样就可以像 45、47 和 50 行那样,作为函数调用时的参数使用。

程序清单:
\lstinputlisting[caption=Go 进程,numbers=right]{ex-communication/src/proc.go}
\end{Answer}


\cleardoublepage
\section{Answers}
\shipoutAnswer
