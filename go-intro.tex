\epi{I'm always delighted by the light touch and stillness of
early programming languages.  Not much text; a lot gets
done. Old programs read like quiet conversations
between a well-spoken research worker and a well-
studied mechanical colleague, not as a debate with a
compiler.  Who'd have guessed sophistication bought
such noise?}{Dick Gabriel}
%%\epi{"I am interested in this and hope to do something."}
%%{Ken Thompson on adding complex numbers to Go}\noindent
This is an introduction into the Go language from Google. Its aim
is to provide a guide to this new and innovative language. As with
learning new things, the best way to do this is to discover it for
yourself. Thus each chapter includes a number of exercises (and answers)
to make yourself acquainted with the language.
Its intended audience is people who are familiar with programming
and know different languages, be it C\cite{c}, C$^{++}$\cite{c++}, 
Java \cite{java}, Erlang\cite{erlang}, Scala\cite{scala} or
Haskell\cite{haskell}. This is \emph{not} a book which teaches you how to 
program, this is a book that teaches you how to use Go.

\begin{lbar}
\noindent The Go language is a young language and
features are still added or even \emph{removed}. Whenever
possible there will be a warning in the margin saying that
something is a new feature. This will be accompanied with
the Go version where this was included first, as in
\texttt{\tiny{}Go release.2010-02-04}.
\end{lbar}

%%\gomarginpar{This a new feature in Go.}
Each chapter concludes with a number of exercises. An exercise
is numbered as \textbf{Q$n$}, where $n$ is a number. After the
exercise number another number in parentheses displays the difficulty
of this particular assignment. This difficulty ranges from 0 to 9, where
0 is easy and 9 is extremely difficult (or at least it should be).
Then a short name is given, for easier reference.
An example is:
\begin{verse}
\textbf{Q1}. (4) A map function \ldots
\end{verse}
    
\noindent gives a question numbered \textbf{Q1} of a level 4 difficulty, concerning a
\func{map()}-function. The answers are included after the exercises on a
new page.

The numbering and setup of the answer is identical to the
exercises, except that an anwer starts with \textbf{A$n$}, where the
number $n$ corresponds with the number of the exercise.

All the included source code (either Go or shell) is tested
and should compile and run. The following convention is used throughout
this book:
\begin{itemize}
\item Code is displayed in \prog{DejaVu mono};
\item Keywords are displayed in \key{DejaVu mono bold};
\item Comments are displayed in \rem{DejaVu mono italic};
\item Lines are numbered. Long lines are continued on the 
next line and don't have a new number;
\item Shell examples use a \pr{} as prompt. Shell snippets don't 
have line numbers.
\end{itemize}

\section{Official documentation}
The reader is assumed to have read (and somewhat understand), the
Go Tutorial \cite{go_tutorial}, and the Effective Go
document \cite{effective_go}. The
website \url{http://golang.org/doc/} is also a very good starting point
for reading up on Go.\footnote{\url{http://golang.org/doc/} itself is served by 
a Go program called \prog{godoc}.}

\begin{lbar}
\noindent Didn't read those documents? Go read them now and stop wasting your time
with this one.
\end{lbar}

\section{Getting Go}
%% Set all Go variables
% $GOOS=linux|darwin|freebsd
% $GOARCH=386|amd64
% $GOROOT
% $GOBIN
The are currently (February 2010) no packages for Go in any Linux
distributions. The route to install Go is thus slightly longer than
it should be. When Go stablizes this situation will change. For now
you need to retrieve the code from the mercurial archive and compile
Go yourself.
\begin{itemize}
\item Install Mercurial (to get the \prog{hg} command), in
Ubuntu/Debian/Fedora you must install the \prog{mercurial} package;

\item For building Go you also need the packages: \prog{bison},
\prog{gcc}, \prog{libc6-dev}, \prog{ed}, \prog{gawk} and \prog{make};

\item Then retrieve the Go source code:
\begin{alltt}
\pr export GOROOT=~/go
\pr hg clone -r release https://go.googlecode.com/hg/ $GOROOT
\end{alltt}
This will install Go in a directory named \dir{go} in your
home directory;

\item Compile Go
\begin{alltt}
\pr cd $GOROOT/src
\pr ./all.bash
\end{alltt}

\end{itemize}

%% TODO make debian package
%% Casts in GO

\section{Compiling and running code}
This is a very concise instruction on how to get your code
compiled and get it running.
The shortest program you can write in Go looks like
this:
\lstinputlisting[label=short,caption=Shortest Go program]{src/short.go}
In which we define the \key{package} \package{main} and supply it with one
function also called \func{main()}. The fully qualified function
name is \func{main.main()}. This is the function that is called
first.  To compile we do the following
\begin{alltt}
\pr 8g short.go \rem{# compiles to short.8 (for 32 bit Intel)}
\pr 8l -o short short.8
\end{alltt}
For 64 bit Intel you should use \prog{6g} and \prog{6l}, this will
generate \prog{short.6}.
You can then execute the program \prog{short}, which of course
does absolute nothing
\begin{alltt}
\pr ./short
\end{alltt}

\section{Online documentation}
Go comes with itsown  documentation in the form of a Go program called
\prog{godoc}. This program \emph{also} powers
\url{http://golang.org/doc}, but you can use it yourself to look
in the online (i.e. installed on your system) documentation. For
instance, suppose, we want to know more about the package \package{hash} 
(more on packages see section \ref{sec:packages} on page
\pageref{sec:packages}). We could then give the command
\prog{godoc hash}, which will display
\begin{alltt}
PACKAGE

package hash
import "hash"

TYPES

type Hash interface \{
    // Write adds more data to the running ha \ldots
\end{alltt}
How to read this and how you can create your own package documentation
is told in chapter \ref{chap:packages}.

\section{Exercises}
\begin{Exercise}[title={Fibonaci},difficulty=5]
\label{ex:fibonaci}
The Fibonaci sequence starts as follows: $1, 1, 2, 3, 5, 8, 13, \ldots$
Or in mathematical terms: $ x_1 = 1; x_2 = 1; x_n = x_{n-1} +
x_{n-2}\quad\forall n > 2 $.

\Question\label{ex:fibonaci q1}Write a function that takes an \type{int} value and gives to
Fibonaci sequence up to that value.

\end{Exercise}

\begin{Answer}
\Question
The following program calculates the Fibonaci numbers.
\lstinputlisting[label=src:fib,caption=A Fibonaci function in Go]{ex-intro/src/fib.go}
\end{Answer}




TODO, compile code, lookup documentation

\cleardoublepage
\section{Answers}
\shipoutAnswer
