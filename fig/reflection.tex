\begin{lstlisting}[caption=使用反射自省,label=src:introspection]
|\begin{tikzpicture}[overlay]
\ubrace{3.2,-5.2}{2.2,-5.2}{%
We are dealing with a \type{PtrValue} and according %
to the documentation\footnote{\texttt{godoc reflect}}:%
\begin{quote} %
Elem 返回 v 指向的值。%
如果 v 是空指针,Elem 返回空值。%
\end{quote} %
同样的在 \var{t} 使用 \func{Elem()} 得到了指针指向的值。}
%%We can use \func{Elem()} to get the type the pointer points to. %
%%In this case \type{*reflect.StructValue}. We have also %
%%used \type{reflect.NewValue(i)} for the type assertion, so that %
%%we get back types in the \type{*reflect} namespace;}
%
\ubrace{4.3,-5.2}{3.6,-5.2}{在 \var{Value} 使用函数 \func{Type()} %
返回 \type{reflect.Type}。需要获取类型的原因是因为那是标签定义的地方;}%
%
\ubrace{7.7,-5.2}{4.8,-5.2}{%
这样获得了 \var{reflect.Type}:\\%
\begin{quote} %
\ldots 返回接口类型 \type{Type} 的对象。包含了指向 \type{*StructType}、\type{*IntType}等类型结构的指针。%
描述了底层类型的细节信息。type switch 或者类型断言可以展示的。%
\end{quote} %
因此可以访问结构中特殊类型的成员。通过 \type{(*reflect.StructType)} 实现;}
%
\ubrace{8.9,-5.2}{7.9,-5.2}{%
\type{StructType} 有若干方法,其中一个是 \func{Field($n$)},返回结构的第 $n^{th}$ 个字段。%
这个返回的类型是 \type{StructField};}
%
\ubrace{9.6,-5.2}{9.1,-5.2}{%
结构 \type{StructField} 有成员 \var{Tag},返回字符串类型的标签名。%
因此,在第 $0^{th}$ 个字段上可以用 \func{.Tag} 访问这个名字:\texttt{Field(0).Tag}。%
这\emph{最终}给出\texttt{namestr}。}
\end{tikzpicture}|
type Person struct {
    name string "namestr" |\coderemark{\texttt{"namestr"} 是标签}|
    age  int
}

p1 := new(Person)   |\coderemark{\func{new} 返回 Person 的指针}|
ShowTag(p1)	    |\coderemark{调用 \func{ShowTag()} 并传递指针}|

func ShowTag(i interface{}) {
 switch t := reflect.NewValue(i).(type) { |\coderemark{在 \type{reflect} 值上的类型断言}|
 case *reflect.PtrValue:	  |\coderemark{因此是 \var{*reflect.PtrValue}}|
	tag := t.Elem().Type().(||*reflect.StructType).Field(0).Tag
||
\end{lstlisting}
Elem 返回 v 指向的值。%
如果 v 是空指针,Elem 返回空值。%
