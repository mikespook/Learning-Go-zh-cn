\begin{Exercise}[title={Strings},difficulty=1]
\label{ex:strings}
\Question \label{ex:strings q1} Create a Go program that prints
the following (up to 100 characters):
\begin{alltt}
A
AA
AAA
AAAA
AAAAA
AAAAAA
AAAAAAA
\ldots
\end{alltt}


\Question \label{ex:strings q2} Create a program that counts
the numbers of characters/runes in this string:
\begin{alltt}
asSASA ddd dsjkdsjs dk
\end{alltt}
Make it also output the number of bytes in that string.
\emph{Hint.} Check out the \package{utf8} package.

\Question \label{ex:string q3} Extend the program from
the previous question to replace the three runes at
position 4 with 'abc'.

\Question \label{ex:string q4} Write a Go program
that reverses a string, so "foobar" is printed as "raboof".

\end{Exercise}

\begin{Answer}

\Question The following program is an answer to the first question.
\lstinputlisting[label=string1,caption=Strings]{ex-basics/src/string1.go}

\Question To answer this question we need some help of
the \package{utf8} package. First we check the documentation
with \prog{godoc utf8 | less}. When we read the documentation
we notice \lstinline{func RuneCount(p []byte) int}. Secondly
we can convert \emph{string} to a \type{byte} slice with
\begin{lstlisting}
str := "hello"
b   := []byte(str)  |\coderemark{A conversion, see section %
\ref{sec:conversions} on page \pageref{sec:conversions}}|
\end{lstlisting}

Putting this together leads to the following program.
\lstinputlisting[label=string2,caption=Runes in strings]{ex-basics/src/string2.go}

\Question Reversing a string can be done as follows:
\begin{lstlisting}
package main

import "fmt"

func main() {
    a := "my string"
    b := ""
    for _,k := range a {
	b = k + b
    }
    fmt.Printf("%s\n", b)
}
\end{lstlisting}

\end{Answer}
