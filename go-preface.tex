\epi{""}
{\textit{}\\ \textsc{}}

\section{Audience}
\noindent{}This is an introduction to the Go language from Google. Its aim
is to provide a guide to this new and innovative language. 

The intended audience of this book is people who are familiar with programming
and know multiple programming languages, be it C\cite{c}, C++\cite{c++}, 
Perl \cite{perl}, Java \cite{java}, Erlang\cite{erlang}, Scala\cite{scala} or
Haskell\cite{haskell}. This is \emph{not} a book which teaches you how to 
program, this is a book that just teaches you how to use Go.

As with
learning new things, probably the best way to do this is to discover it for
yourself by creating your own programs.
Each chapter therefor includes a number of exercises (and answers)
to acquaint you with the language.
An exercise
is numbered as \textbf{Q$n$}, where $n$ is a number. After the
exercise number another number in parentheses displays the difficulty
of this particular assignment. This difficulty ranges from 0 to
9, where 0 is easy and 9 is difficult. 
Then a short name is given, for easier reference.
For example:
\begin{verse}
\textbf{Q1}. (1) A map function \ldots
\end{verse}
introduces a question numbered \textbf{Q1} of a level 1 difficulty, concerning a
\func{map()}-function. The answers are included after the exercises on a
new page.
The numbering and setup of the answers is identical to the
exercises, except that an answer starts with \textbf{A$n$}, where the
number $n$ corresponds with the number of the exercise.

\section{Book layout}
Chapter \ref{chap:intro} gives a short introduction and history of Go. It also
tells how to get the source code of Go itself. It assumes a Unix-like environment.

Chapter \ref{chap:basics} tells about the basic types, variables and control
structures available in the language.

In the third chapter we look at functions, that basic building blocks of Go programs.

In chapter \ref{chap:packages} we see that function and data can be grouped together
in packages. Packages provide namespaces.

5: beyond the basics, create new types.

Go doesn't do Object Orientation, instead it has interfaces. In chapter \ref{chap:interfaces}
we will show how this works.

With the go keyword function can be started in separate routines (called goroutines). 
Communication with those goroutines is done via channels.

The last chapter show how to interface with the rest of the world from within 
a Go program. How create files and read and wrote from and to them. We also briefly
look into networking.
