\begin{Exercise}[title={方法调用},difficulty=8]
\label{ex:methodcalls}
\Question \label{ex:methodcalls q1} 假设有下面的程序:
\begin{lstlisting}
package main

import "container/vector"

func main() {
	k1 := vector.IntVector{}
	k2 := &vector.IntVector{}
	k3 := new(vector.IntVector)
	k1.Push(2)
	k2.Push(3)
	k3.Push(4)
}
\end{lstlisting}
\var{k1},\var{k2} 和 \var{k3} 的类型是什么?

\Question 当前,这个程序可以编译并且运行良好。在不同类型的变量上 \func{Push}
都可以工作。\func{Push} 的文档这样描述:
\begin{quote}
func (p *IntVector) Push(x int)
Push 增加 x 到向量的末尾。
\end{quote}
那么接受者应当是 \type{*IntVector} 类型,为什么上面的代码可以工作?

\end{Exercise}

\begin{Answer}
\Question \var{k1} 的类型是 \type{vector.IntVector}。为什么?
这里使用了符号 \verb|{}|,因此获得了类型的值。
变量 \var{k2} 是 \type{*vector.IntVector},因为获得了复合语句的地址(\verb|&|)。
而最后的 \var{k3} 同样是 \type{*vector.IntVector} 类型,因为 \func{new}
返回该类型的指针。

\Question 在 \cite{go_spec} 的“调用”章节,有这样的描述:
\begin{quote}
当 \var{x} 的方法集合包含 \func{m},
并且参数列表可以赋值给 \func{m} 的参数,方法调用 \func{x.m()} 是合法的。
如果 \var{x} 可以被地址化,而 \var{\&x} 的方法集合包含 \func{m},
\func{x.m()} 可以作为 \func{(\&x).m()} 的省略写法。
\end{quote}
换句话说,由于 \var{k1} 可以被地址化,而 \type{*vector.IntVector}
\emph{具有} \func{Push} 方法,调用 \lstinline{k1.Push(2)} 被 Go 转换为 
\lstinline{(&k1).Push(2)} 来另类型系统愉悦(也另你愉悦——现在你了解到这一点)。
\footnote{参阅本章的第 "\titleref{sec:methods}" 节。}

\end{Answer}
