\begin{Exercise}[title={Stack as package},difficulty=2]
\label{ex:stack-package}
\Question\label{ex:stack-package q1} 
See the Q\ref{ex:stack} exercise. In this exercise we want to create
a separate package for that code.
Create a proper package for your
stack implementation, \func{Push}, \func{Pop} and the \type{Stack} type need to be
exported.

\Question\label{ex:stack-package q2} Write a simple unit test for this package.
You should at least test that a \func{Pop} works after a \func{Push}.

\end{Exercise}

\begin{Answer}
\Question There are a few details that should be changed to make a proper package
for our stack. First, the exported functions should begin with a capital 
letter and so should \type{Stack}. The package file is named \file{stack-as-package.go}
and contains:
\lstinputlisting[caption=Stack in a package]{ex-packages/src/stack-as-package.go}

\Question To make the unit testing work properly you need to do some
preparations. We'll come to those in a minute. First the actual unit test.
Create a file with the name \file{pushpop\_test.go}, with the following contents:
\lstinputlisting[caption=Push/Pop test]{ex-packages/src/pushpop_test.go}

For \prog{go test} to work we need to put our package files in a directory
under \var{\$GOPATH/src} (see page \pageref{"sec:settings used"}).

\begin{display}
\pr \user{mkdir $GOPATH/src/stack}
\pr \user{cp pushpop_test.go $GOPATH/src/stack}
\pr \user{cp stack-as-package.go $GOPATH/src/stack}
\end{display}

\begin{display}
\pr \user{go test stack}
ok      stack   0.001s
\end{display}
\end{Answer}
