\begin{Exercise}[title={Fibonaci II},difficulty=7]
\label{ex:fibonaci II}
\Question\label{ex:fibonaci II q1}
This is the same exercise as the one given page \pageref{ex:fibonaci} 
in exercise \ref{ex:fibonaci}. For completeness the complete question:

\begin{quote}
The Fibonaci sequence starts as follows: $1, 1, 2, 3, 5, 8, 13, \ldots$
Or in mathematical terms: $ x_1 = 1; x_2 = 1; x_n = x_{n-1} +
x_{n-2}\quad\forall n > 2 $.

Write a function that takes an \type{int} value and gives 
that many terms of the Fibonaci sequence.
\end{quote}

\begin{lbar}
\emph{But} now the twist: You must use channels.
\end{lbar}


\end{Exercise}

\begin{Answer}
\Question
The following program calculates the Fibonaci numbers using channels.
\lstinputlisting[label=src:fib II,caption=A Fibonaci function in Go]{ex-channels/src/fib.go}
\end{Answer}


