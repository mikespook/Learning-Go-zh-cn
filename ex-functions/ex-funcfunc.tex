\begin{Exercise}[title={Functions that return functions},difficulty=4]
\label{ex:function}
\Question\label{ex:function q1} Write a function that returns a function
that performs a $+2$ on integers. Name the function \func{plusTwo}.
You should then be able do the following:
\begin{lstlisting}
p := plusTwo()
fmt.Printf("%v\n", p(2))
\end{lstlisting}
Which should print 4.

\Question\label{ex:function q2} Generalize the function from \ref{ex:function q1},
and create a \func{plusX(x)} which returns a functions that add \var{x} to an
integer.
\end{Exercise}

\begin{Answer}
\Question
\begin{lstlisting}
func main() {
        p2 := plusTwo()
        fmt.Printf("%v\n",p2(2))
}

func plusTwo() func(int) int { |\longremark{Define a new function that returns a function. %
See how you you can just write down what you mean;}|
        return func(x int) int { return x + 2 } |\longremark{Function literals at work, %
we define the +2--function right there in the return statement.}|
}
\end{lstlisting}
\showremarks

\Question
\begin{lstlisting}
func plusX(x int) func(int) int {
        return func(y int) int { return x + y } 
}
\end{lstlisting}
\end{Answer}
