\begin{Exercise}[title={Documentation},difficulty=1]
\label{ex:doc}
\Question
Go's documentation can be read with the \prog{godoc} program, which is
included the Go distribution.

\prog{godoc hash} gives information about the \package{hash} package:
\vskip\baselineskip
\begin{display}
\pr \user{godoc hash}
PACKAGE

package hash

...
...
...

SUBDIRECTORIES

        adler32
        crc32
        crc64
        fnv

\end{display}
\vskip\baselineskip
With which \prog{godoc} command can you read the documentation of \package{fnv} contained in
\package{hash}?

\end{Exercise}

\begin{Answer}
\Question
The package \package{fnv} is in a \emph{subdirectory} of
\package{hash}, so you will only need\quad \texttt{godoc hash/fnv}. Also \package{hash/fnv} is an
official (included in Go 1) package, so we should use \prog{godoc}, not \prog{go doc}.

Specific functions inside the ``Go manual'' can also be accessed. For
instance the function \func{Printf} is described in \package{fmt}, but to
only view the documentation concerning this function use: \prog{godoc fmt Printf}{} .

All the built-in functions are also accesible by using \prog{godoc}: \prog{godoc builtin}.
\end{Answer}
