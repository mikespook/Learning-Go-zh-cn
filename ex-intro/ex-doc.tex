\begin{Exercise}[title={Documentation},difficulty=1]
\label{ex:doc}
\Question
Go's documentation can be read with the \prog{godoc} program, which is
included the Go distribution.

\prog{godoc hash} gives information about the \package{hash} package. Reading the
documentation on \package{compress} gives the following result:
\vskip\baselineskip
\begin{display}
\pr \user{godoc compress}
SUBDIRECTORIES

        bzip2
        flate
        gzip
        lzw
        testdata
        zlib
\end{display}
\vskip\baselineskip
With which \prog{godoc} command can you read the documentation of \package{gzip} contained in
\package{compress}?

\end{Exercise}

\begin{Answer}
\Question
The package \package{gzip} is in a \emph{subdirectory} of
\package{compress}, so you will only need\quad \texttt{godoc compress/gzip}.

Specific functions inside the "Go manual" can also be accessed. For
instance the function \func{Printf} is described in \package{fmt}, but to
only view the documentation concerning this function use: \prog{godoc fmt Printf}{} .

You can even display the source code with: \prog{godoc -src fmt Printf} .     

All the built-in function are also accesible by using godoc: \prog{godoc builtin}.
\end{Answer}
